\chapter {Software}

\section{Spreadsheets}

\begin{description}
  \item[\href{https://advait.org/files/sarkar_2018_calcview.pdf}{Calculation View: multiple-representation editing in spreadsheets}] {\cite{sarkar2018calculation}.
    \\ Presents "Calculation View" (CV), a novel alternate view in Microsoft Excel. Allows users to edit spreadsheets in a more high-level way. Introduces ranges, named ranges, pseudocells, and a block detection algorithm. Details further work to expand CV view, including text interface and expanded features.
    \begin{itemize}
      \item \question{Can CV store intermediate values that the user might not need represented on the sheet?}
      \item \question{Can CV be used on its own? Is there a use case for CV being used on its own?}
    \end{itemize}
    }
  \item[\href{https://www.microsoft.com/en-us/research/uploads/prod/2018/11/elastic-sdfs-5bfd1d48e9dd2.pdf}{Elastic Sheet-Defined Functions: Generalising Spreadsheet Functions to Variable-Size Input Arrays}] {\cite{elastic-sheet-defined-functions-generalising-spreadsheet-functions-to-variable-size-input-arrays}.
    \\ Defines formal syntax and semantics for generalising spreadsheet functions to variable-size input arrays (ranges). Outlines algorithm for identifying generalizations and interpreting most likely generalizations given multiple options.
    }
\end{description}