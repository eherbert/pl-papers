\chapter {Systems History}

\section {Template to Copy}

\begin{description}
    \item[\textbf{Summary}]
    \item[\textbf{Context}]
    \item[\textbf{Discussion Points}]
    \item[\textbf{Interest Factor}]
    \item[\textbf{Significance}]
    \item[\textbf{Personal Assessment}]
\end{description}

\section {The Education of a Computer \cite{hopper1952education}}

\begin{description}
    \item[\textbf{Summary}] This paper goes through iterations of "educating" a computer, to where the mathematician using a computer gets the boot, the computer becomes the mathematician, and the programmer becomes an integral part of the computer.
        A subroutine is a function-like that performs some computation. It has an entry line, exit line, result line, argument lines, and routine lines.
        Different procedures interact with different lines of the subroutines.
        The computer is given a bunch of smaller mathematical subroutines that the programmer can use to help construct their programs.
        It is hypothesised that more "subroutines" could be developed and combined.
        It is unclear to me what the correlation is between subroutines and modern day functions.
    \item[\textbf{Context}] I think this work is difficult for modern day programmers to understand/ fathom (myself included).
        It is difficult to think of what may lie between assembly-style jumps and Haskell, for example, which is the space that this paper explores.
    \item[\textbf{Discussion Points}] (1) Page 245 discusses turning using programs that contain subroutines as a subroutines itself, which I would consider a profound observation.
        Yet the "conclusion" focuses just on the arithmetic type mathematical advanecs. Hmm.
        (2) The proposed UNIVAC is claimed to "not forget" and "not make mistakes."
        We all know that that is not the case.
\end{description}