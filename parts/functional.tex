\chapter {Functional Programming}

\section{Data Structures}

\begin{description}
    \item[\href{https://www.cs.ox.ac.uk/files/3378/PRG56.pdf}{An Introduction to the Theory of Lists}] {\cite{bird1987introduction}.
          \\ Formal definitions for lists and list operations. Discusses common list operations (map, filter, reduce, etc.) and offers best use cases. Provides examples of multiple interacting operations and operation equivalences.
          \\ \question{What are infinite lists?}}
    \item[\href{https://www.st.cs.uni-saarland.de/edu/seminare/2005/advanced-fp/docs/huet-zipper.pdf}{Functional Pearl: The Zipper}] {\cite{huet1997zipper}.
          \\ Describes a data structure that is akin to a zipper, for use in situations in which trees need to be modified non-destructively. Handles can be on particular elements of the tree, where locations hold the downward current subtree and the upward path. Functions are given for navigation left, right, up, and down, retrieving the nth element, changing an element in place, inserting elements left, right, and up, and deleting elements. A memoization approach is suggested, where "scars" hold tree structure for frequently visited elements.
          \\ Scala implementation at \href{https://github.com/stanch/zipper}{https://github.com/stanch/zipper}
          \\ \question{Binary trees are shown, but binary tree example doesn't allow for any data stored in the tree?}}
\end{description}
