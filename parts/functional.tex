\chapter {Functional Programming}

\section {An Introduction to the Theory of Lists \cite{bird1987introduction}}
Formal definitions for lists and list operations. Discusses common list
operations (map, filter, reduce, etc.) and offers best use cases. Provides
examples of multiple interacting operations and operation equivalences.
\begin{itemize}
    \item \question{What are infinite lists?}
\end{itemize}

\section {Functional Pearl: The Zipper \cite{huet1997zipper}}
Describes a data structure that is akin to a zipper, for use in situations in
which trees need to be modified non-destructively. Handles can be on particular
elements of the tree, where locations hold the downward current subtree and the
upward path. Functions are given for navigation left, right, up, and down,
retrieving the nth element, changing an element in place, inserting elements
left, right, and up, and deleting elements. A memoization approach is suggested,
where "scars" hold tree structure for frequently visited elements.
\begin{itemize}
    \item Scala implementation at \href{https://github.com/stanch/zipper}{https://github.com/stanch/zipper}.
    \item \question{Binary trees are shown, but binary tree example doesn't
    allow for any data stored in the tree?}
\end{itemize}

\section{Monads for functional programming \cite{wadler1995monads}}
Monads might be thought of as a way to add impure or side-effect behavior to
pure languages (Haskell, etc.). If one has a function of type $a \rightarrow b$
(a function from $a$ to $b$), it can be modified to include side effects by
$a \rightarrow M b$ (a function from $a$ to $b$, with a possible additional
effect captured by $M$.) A monad $(M, \mathit{unit}, \star)$ has three
components:

\begin{enumerate}
    \item Type constructor $M$
    \item Operation $\mathit{unit}$ that turns a value into the computation that
    returns that value and does nothing else:
    \( \mathit{unit} :: a \rightarrow M a \)
    \item Operation $\star$ which allows us to apply a function of type
    $a \rightarrow M b$ to a computation of type $M a$.
    \( \star :: M a \rightarrow (a \rightarrow M b)\)
\end{enumerate}

You can write a monad as $m \star \lambda a . n$. The form $\lambda a . n$ is a
lambda expression, where $a$ is scoped within $n$. This can be read as, perform
computation $m$, then bind $a$ to the result, such that $a$ is then scoped
within $n$. In other words ``\lstinline|let a = m in n|.'' Monads also have
several properties:

\begin{enumerate}
    \item[Left unit] $\textit{unit } a \star \lambda b . n = n[a/b]$. Which is
    to say, Compute $a$ and bind the result to $b$ in $n$. This is equivalent to
    evaluating $n$ with all instances of $a$ substituted for $b$.
    \item[Right unit] $m \star \lambda a . \textit{unit } a = m$. Which is to
    say, compute $m$ and bind the result to $a$ and return $a$.
    \item[Associative] $m \star (\lambda a . n \star \lambda b . o) =
    (m \star \lambda a . n) \star \lambda b . o$
\end{enumerate}

``A binary operation with left and right unit that is associative is called a
monoid. A monad differs from a monoid in that the right operand involves a
binding operation.''